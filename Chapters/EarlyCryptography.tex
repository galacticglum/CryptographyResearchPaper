\chapter{Early Cryptography} % Main chapter title
\label{Early Cryptography} % For referencing the chapter elsewhere, use \ref{Chapter1} 

%----------------------------------------------------------------------------------------

As discussed in the \hyperref[Introduction]{Introduction} chapter, cryptography has a very long and interesting history. 
In this chapter will dive further and explore the various cryptography methods and inventions which paved the 
way for modern-day cryptography.

\section{Greek Cryptography}

\subsection{The Scytale}

You may have heard of the \textit{Spartan} warrior society, famed for it's military-focused lifestyle featuring austerity,
strength, and fitness \cite{wiki:spartan_army}. The Spartan warrior society also utilized a cryptography device called the
\textit{Scytale}. The Scytale is a cryptography device in the shape of cylinder which would be distributed to both the sender
and receiver. In order to encrypt a message, a narrow strip of parchment paper would be wrapped around and then the message
written on the strip of parchment paper. Once unwrapped, the paper displays seemingly useless and confusing letters. The 
strip of paper containing the encrypted message would then be sent to whom it is intended for. To decrypt, the receiver would
wrap the strip of paper around a Scytale with the same diameter as the one used to encrypt the message 
\cite{history_of_cryptography}.

This type of cryptography technique is commonly referred to as a \textit{transposition} cipher because the letters
in the message remain the same while the position of the letters change \cite{wiki:scytale}. In mathematics, to encrypt
(using the transposition cipher) you would use a \textit{bijective} function and to decrypt you would use an 
\textit{inverse function}. The Scytale allowed the ancient Greeks (and Spartans in particular) to send messages 
securely. Though it should be said that this method of encryption is very easy to break by simply brute-forcing the different 
diameters.

\subsection{Caesar Cipher}

Named after the Greek emperor \textit{Julius Caesar}, this cryptography method works by substituting the $ith$ letter in the 
message by the letter that is $x$ positions to the right. Then to decrypt, you substitute the $ith$ letter in the message by 
the letter that is $x$ positions to the left. Caesar Cipher is commonly referred to an \textit{addition} or \textit{substitution} 
cipher.

\section{Cryptography in the Medieval and Renaissance Periods}

For a long time, advancements in cryptography were put to a halt. Even though the Romans were aware and utilized Greek 
cryptography techniques, there was no focus on inventing new techniques. That is until the \textit{Medieval and Renaissance}
periods.

\subsection{Arabic Cryptography}

Modern cryptography techniques originated from the Arabic people, the first to document cryptanalysis techniques 
in a systemic fashion. During 800 AD, the Arab mathematician \textit{Al-kindi} invented the \textit{frequency-analysis} 
technique, this technique had so much impact on the field of cryptography that it was the most significant advance in
cryptography until World War 2. In Al-kindi's \textit{Risalah fi Istikhraj al-Mu'amma} 
(\textit{Manuscript for the Deciphering Cryptographic Messages}) he described the first techniques of cryptanalysis,
classification of ciphers, and most importantly gave the first descriptions of frequency-analysis .
In addition, The \textit{Subh al-a 'sha}, a 14-volume encyclopedia written by \textit{Ahmad al-Qalqashandi} included a dedicated section
on cryptography \cite{wiki:history_of_cryptography}.
 
\subsection{Cryptography in Europe}

In Europe, cryptography became a more integral part of defense due to political and religious change. During and after the Renaissance period,
many contributions to the field of cryptography can be attributed to the citizens of Italian states \cite{wiki:history_of_cryptography}.
Furthermore, during the reign of Queen Elizabeth I cryptanalysts were able to decrypt messages regarding a plot to assassinate the Queen.
This plot is known as the \textit{Babington Plot}.\footnote{In 1586, Anthony Babington and John Ballard colluded to assassinate Queen Elizabeth I 
and put \textit{Mary, Queen of Scots} on the throne. After cryptanalysts decrypted a message sent by Queen Mary herself regarding the assassination, Queen
Elizabeth I ordered the execution of Queen Mary and shortly after, she was executed \cite{wiki:babington_plot}. }

\section{Conclusion}

As you can see, early cryptography laid out the fundamentals for core techniques used in modern-day cryptography.
Both the \textit{transposition} and \textit{substitution} ciphers remain fundamental concepts in complex cryptography
algorithms used today. In addition, the everlasting contributions made by mathematician Al-kindi in the field
of cryptography were so important that no other invention exceeded Al-kindi's until World War 2. In addition, his descriptions of 
frequency-analysis tremendously aided cryptographers in cryptanalysis; such as preventing the assassination of Queen of Elizabeth I.
