\chapter{Introduction} % Main chapter title
\label{Introduction} % For referencing the chapter elsewhere, use \ref{Chapter1} 

%----------------------------------------------------------------------------------------

% % Define some commands to keep the formatting separated from the content 
% \newcommand{\keyword}[1]{\textbf{#1}}
% \newcommand{\tabhead}[1]{\textbf{#1}}
% \newcommand{\code}[1]{\texttt{#1}}
% \newcommand{\file}[1]{\texttt{\bfseries#1}}
% \newcommand{\option}[1]{\texttt{\itshape#1}}

%----------------------------------------------------------------------------------------

Cryptography has been rooted in some of history's most important conflicts. 
During World War 1, in January of 1917, British cryptanalysts deciphered a German
telegram from the German Foreign Minister \textit{Arthur Zimmermann} to the German Minister
of Mexico: \textit{Heinrich von Eckardt}. The message was a proposition to Mexico, offering
United States territory in return for joining the Central Powers.\footnote{The Central Powers 
were comprised of Germany, Austria-Hungary, the Ottoman Empire and Bulguria; also known as
the \textit{Quadruple Alliance} \cite{wiki:central_powers}.} On February 24th 1917, the British
presented the telegram to President \textit{Woodrow Wilson} of the United States of America. 
Shortly after, on April 6th, 1917, the United States of America officially declared war
on the Central Powers \cite{the_zimmermann_telegram}. The impact that cryptanalysis had on
the war was tremendous, forever changing the course of history. Not only had it caused the
United States of America to enter the war but it also put cryptography at the forefront of political
and military attention. 

The history of cryptography can be traced back to 4000 years ago in the Egyptian town of 
\textit{Menat Khufu}. Hieroglyphics on the tomb of \textit{Khnumhotep II} were written with unusual 
symbols such as to obfuscate the meaning \cite{history_of_cryptography}. 

The word \textit{cryptography} derives from the Greek word \textbf{\textit{kryptos}}
meaning hidden and \textbf{\textit{graphein}} meaning writing \cite{pawlan_cryptography}.
Cryptography is the science of concealing information in such a way that the message
may only be read by whom it is intended for \cite{cryptography_definition}. 
Likewise, \textit{cryptanalysis} is the study of deciphering codes where the key is unknown 
\cite{cryptanalysis_definition}.  

A live chat application is a software which allows two or more users to communicate in real-time. 
Over the internet, the messages that are sent will typically be relayed by many other hosts
outside of the control of the sender or recipient. Because of this, it is possible for data
being transmitted over the network to be compromised by a third-party. To combat this, we can employ
the use of cryptography to encrypt messages before being sent and then decrypt them upon delivery. 
This paper will examine different cryptography algorithms and how they work within a live chat 
application.  
