\chapter{Symmetric-key Cryptography} % Main chapter title
\label{Symmetric-key Cryptography} % For referencing the chapter elsewhere, use \ref{Chapter1} 

%----------------------------------------------------------------------------------------

\textit{Symmetric-key} cryptography (also commonly referred to as \textit{private-key cryptography}) is a system of cryptography which
uses the same keys for both encryption of plaintext and decryption of ciphertext. The keys must be shared between the different
parties accessing and handling the information. The requirement that all (authorized) parties must have access to the secret key
is the biggest drawback of symmetric-key cryptography \cite{wiki:symmetric_key_cryptography}.     

This chapter will discuss and implement the different types of symmetric-key cryptography algorithms. 
First we will explore and implement the various classical cryptography examples (some of which are discussed in the 
\hyperref[Early Cryptography]{Early Cryptography} chapter). From there, we will continue and take a look at more complex modern-day algorithms 
and discuss their implementation. 

\section{Transposition Ciphers}