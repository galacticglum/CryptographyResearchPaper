\chapter{Analysis}
\label{Analysis}

Like any algorithm, we can not define "the best cryptography method;" rather, the cryptography algorithm that you use is 
very case-dependant. In order to find a suitable candidate(s), one must examine the merits of a set of different algorithms.

When it comes to cryptography, security is at the root of the question. But how much do you \textit{really} value security?
For some, it may be a matter of "life or death." For others, security is a secondary feature of their program. Maybe you don't
even care about security? This can surely be the case, consider a puzzle game, one could employ the use of a substitution or 
transposition cipher to create an interesting cryptography puzzle.

Let us consider a \textit{live chat application}. This application will send messages from one user to another (over the network).
We will assume that the messages sent are sensitive. In this case, security does seem to play a vital role in the function
of the application. Though we must also consider the \textit{comprising party}---the party which intends to infiltrate the 
communication network. 

Say we are designing a chat application for an intranet,\footnote{An intranet is a private network that is only accessible to
members of the network. An organization may use an intranet; only staff of the organization may access the network.} there may 
no comprising party or the risk of a comprising party is low. Therefore we may not require any complex cryptography systems, 
instead opting for a simple symmetric-key algorithm (such as Caesar Cipher or the slightly more complex DES scheme). On the 
other hand, if we are developing a chat application for the internet where anyone can access the communication network. We may
place more importance on security as the network is public.

What if do we require a complex asymmetric-key system? We could use the discussed RSA scheme. All around, the RSA
scheme is an excellent cryptography method---it is extremely reliable---it is purely a mathematical solution---and it is secure. 
But as always, there exists a caveat. The RSA scheme generates insanely large numbers. This is due to the fact that our plaintext
must be smaller than our modulus (the shared key). As discussed in 
\hyperref[Asymmetric-key cryptography]{Asymmetric-key cryptography}, if our plaintext is
larger than our modulus, we must split our plaintext into appropriately sized blocks. We then encrypt each block in some mode
of operation (such as Feistel cipher). Unfortunately, this type of encryption will reveal redundancies in the ciphertext which
makes the cipher easier to break. Alternatively, we can expand the size of our modulus (and private and public keys)
to suit the size of our plaintext. This technique will work in theory but in practice there exists a hardware limitation---the
size of a data type; RSA will quickly exceed the max value of a $64$-bit integer. Instead we can use a \textit{Big Integer}, a
data structure which allows us to store \textit{very very} large numbers. This does however use a lot of memory. Nonetheless,
despite the existing alternatives, there exists the fact that RSA is a memory-hog. And despite advancements in computer 
technology, there always exists an instantial upper-bound---whether it be storage or speed.
