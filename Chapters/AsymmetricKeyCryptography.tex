\chapter{Asymmetric-key cryptography}
\label{Asymmetric-key cryptography} % For referencing the chapter elsewhere, use \ref{Chapter1} 

%----------------------------------------------------------------------------------------

\textit{Asymmetric-key} cryptography (also commonly referred to as \textit{public-key} cryptography) is a system of cryptography which uses \textit{pairs} of keys to 
encrypt and decrypt data. The pairs of keys consist of public keys which can be distributed, and private keys which are only known to the owner. Essentially,
asymmetric-key cryptography allows us encrypt and decrypt data where the key for encryption is different than the key for decryption. Knowledge of either
of these keys does not allow us to find the other one \cite{classical_algebra}. The name comes from that fact that we use multiple keys in the process of encryption and decryption
and therefore have \textit{asymmetric} keys.

Suppose, John and Mary want to securely communicate using the scheme above. First, John would produce a encryption key $K_1$ and a decryption key $K_2$. John 
would then send Mary the encryption key $K_1$. Mary can now encrypt data using the key $K_1$ and send the result to John who can decrypt her data using his decryption key
$K_2$. Let's now introduce Bob, who intercepts the transmission of the encryption key $K_1$. Bob can now encrypt messages using the key $K_1$ but is unable to decrypt 
them --- only John can decrypt messages as he is the only one who posses the decryption key $K_2$. If John also wants to send messages to Mary, Mary will have to create
an encryption key $x$ and a decryption key $y$ and then send the encryption key $x$ to John. Now, John can encrypt his messages using the encryption key $x$ (sent by Mary)
and then send the encrypted data to Mary where she will decrypt the message using the decryption key $y$. As you can see, in this scheme, we don't care if a third-party 
(such as Bob) gets a hold of one of the encryption keys --- hence they could just be \textit{public knowledge} \cite{classical_algebra}. 

\section{The Rivest---Shamir---Adleman Scheme}

The \textit{Rivest---Shamir---Adleman} scheme (or \textit{RSA} for short) is one of the first practical asymmetric-key cryptosystems \cite{classical_algebra}. 
The cryptosystem is named after it's discoverers \textit{Ron Rivest}, \textit{Adi Shamir}, and \textit{Len Adleman}, who gave a definition on how 
asymmetric-key cryptography could be realized \cite{classical_algebra}.

The RSA scheme involves four distinct steps: \textit{key generation}, \textit{key distribution}, \textit{encryption}, and \textit{decryption}

\subsection{Key Generation}

A user who wishes to participate in a session of the scheme must first generate two keys: one public key for encryption and one private key for decryption. To do so,
the user will first select two large prime numbers $p$ and $q$, and then multiply them together to calculate the integer $n = pq$. 
If $$\mathlarger{\phi(n) = (p-1)(q-1)}$$ then the user will also select an integer $e > 1$ such that $GCD(e,\phi(n))=1$. 
The integer $e$ will be used for the calculation of the encryption key. The user now solves for the equation: $$\mathlarger{ed \equiv 1 \pmod{\phi(n)}}$$

According to the \textit{Linear Congurnece Theorem}, since $GCD(e,\phi(n))=1$, there is exactly one congruence class modulo $\phi(n)$ which satisfies the aforementioned
congruence -- or in other words, there is only one integer between $0$ and $\phi(n)$ which satisfies the congruence, let $d$ represent this integer 
\cite{classical_algebra}. Now, the pair of integers $(e,n)$ represent the user's public keys, and the pair of integers $(d,n)$ represent the user's private keys. 
In other words, we now have the following keys:

\begin{itemize}
    \item $n$ --- the shared key.
    \item $e$ --- the public encryption key.
    \item $d$ --- the private decryption key.
\end{itemize}

\subsection{Key Distribution}

Suppose that John and Mary are communication using the RSA scheme. John must know Mary's public key $e$ to encrypt his message and Mary must use her private key $d$ to 
decrypt his message. Therefore, Mary must transmit her $(e,n)$ key pair to John via a reliable (though not secret) route. 

\subsection{Encryption and Decryption}

To encrypt the message $M$ --- first, the sender of the message would lookup the recipient's public key pair $(e,n)$ \cite{classical_algebra}. 
Then the sender would compute the cipertext $C$ using the following equation: 
$$\mathlarger{M^e \equiv C \pmod{n} \text{\hspace{1em}where\;\;}0 \leq C < n}$$

The sender will then take the ciphertext $C$ and send it to the recipient.

Decryption is the inverse of the encryption processes. The recipient of the message, will decrypt the message using the same equation for encryption but rather using it's
own private decryption key pair $(d,n)$ \cite{classical_algebra}. Decryption can be expressed as 
$$\mathlarger{C^d \equiv R \pmod{n} \text{\hspace{1em}where\;\;}0 \leq R < n}$$

The decrypted message $R$ is the same as the original message $M$.

It is very important to note that our message $M$ must be smaller than our $n$. In the case that $M$ is greater than $n$, we must split $M$ into blocks which are smaller
than $n$. For example, if $n=12319$ and we want to encrypt the message $06212626$, we would split the message into two blocks --- $0621$ and $2626$, encrypt each block, 
and then finally combine the blocks into one blocks (making sure to maintain the order of the blocks).
